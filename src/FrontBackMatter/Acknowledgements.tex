% !TEX TS-program = pdflatex
% !TEX root = ../tesi.tex

%*******************************************************
% Acknowledgements
%*******************************************************
\pdfbookmark{Acknowledgements}{Acknowledgements}

\chapter*{Riconoscimenti}

\begin{flushright}
\itshape
I videogiochi sono uno strumento che ci permette di entrare in contatto con altri universi. \\
Grazie ai supporti tecnologici è quindi possibile sperimentare qualcosa che va ben oltre il mondo reale. \\
È come quando si legge un libro appassionante, \\
ma lo stesso effetto si può provare quando si ascolta della musica oppure si guarda un film. \\
Allo stesso modo, attraverso i videogiochi si può entrare in contatto con degli universi paralleli, \\
ma soprattutto si può vivere questa esperienza in modo più coinvolgente e personale.
I videogiochi hanno questa forza. \\


\medskip
--- Kazunori Yamauchi

\end{flushright}

\bigskip
\bigskip

\heartpar{I wish first of all to thank the members of the Italian \TeX{} and \LaTeX{} User Group, in particular
Claudio Beccari, Fabiano Busdraghi, Gustavo Cevolani, Rosaria D'Addazio, Agostino De Marco, Massimiliano Dominici, Gloria Faccanoni, Claudio Fiandrino, Heinrich Fleck, Enrico Gregorio, Massimo Guiggiani, Roberto Giacomelli, Gianluca Gorni, Maurizio Himmelmann, Jer\'onimo Leal, Paride Legovini, Lapo Filippo Mori, Gianluca Pignalberi, Luigi Scarso, Marco Stara, Andrea Tonelli, Ivan Valbusa, Emiliano Giovanni Vavassori and Emanuele Vicentini,
for their invaluable aid during the writing of this work, the detailed explanations, the patience and the precision in the suggestions, the supplied solutions, the competence and the kindness: thank you, guys!
Thanks also to all the people who have discussed with me on the forum of the Group, prodigal of precious observations and good advices.
Finally, thanks to Andr\'e Miede, for his wonderful ClassicThesis style, and to Daniel Gottschlag, who gave to me the hint for this original reworking.}
