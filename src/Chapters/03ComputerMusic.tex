% !TEX TS-program = pdflatex
% !TEX root = ../tesi.tex

%************************************************
\chapter{LA COMPUTER MUSIC PER I VIDEO-GAMES}
\label{chp:La Computer Music per i Video-Games}
%************************************************
La computer music è l'applicazione della tecnologia informatica nella composizione musicale, per aiutare i compositori umani a creare nuova musica o per avere computer che creano musica in modo indipendente, come con i programmi di composizione algoritmica. \\
Include la teoria e l'applicazione di tecnologie software per computer nuove ed esistenti e aspetti di base della musica, come la sintesi del suono, l'elaborazione del segnale digitale, il sound design, la diffusione sonora, l'acustica, l'ingegneria elettrica e la psicoacustica.\\
Il campo della computer music può far risalire le sue radici alle origini della musica elettronica e ai primi esperimenti e innovazioni con strumenti elettronici a cavallo del 20 ° secolo.\\
I videogiochi emersero come forma di intrattenimento popolare lungo gli ultimi anni settanta (il periodo delle console di seconda generazione), e la loro musica era registrata su musicassetta o dischi in vinile.\\
Un altro metodo più economico per realizzarla consisteva nell'adoperare un chip, che sostituiva gli impulsi elettrici del codice di un computer a onde sonore analogiche.\\
A partire dal 1980, alcuni videogiochi iniziarono ad adoperare, oltre ai sintetizzatori ed ai computer, apparecchiature digitali e campionamenti per fare musica oggi riconosciuta come chiptune.\\
Mentre le console casalinghe si avvicinavano alla quarta generazione (o era dei 16-bit), quell'approccio ibrido continuò ad essere usato.
Nel 1988 le console offrivano, fra le altre innovazioni, una grafica avanzata ed una sintesi sonora perfezionata rispetto a quelle dei supporti precedenti.\\
 A partire dagli anni novanta, la musica per videogiochi iniziò a presentare sonorità realizzate con strumenti musicali veri e propri.\\
 Questo fu dovuto all'evoluzione dei computer che divennero, grazie alle capacità di contenimento dei dati dei giochi sempre più rapidi e potenti.
 Successivamente arrivò il supporto da 64 bit composto da due coprocessori di 32 bit ciascuno, seguito dal riverbero, per arrivare poi all'invenzione di IMUSE, un motore di gioco che controlla la musica del videogioco in tempo reale.
 Sino all'arrivo della musica su licenza che prevedeva l'introduzione di composizione già esistenti di musiche hip-hop, rap, pop, rock e così via.\\
 Oggi esistono due categorie: Musiche originali e Musica con licenza.

 \subsection*{La Musica Generativa}
 La mia aggiunta personale, non presente attualmente nella storia musicale dei videogiochi, è quella di un ramo della Computer Music,
 ovvero la Musica Generativa.\\
 La musica generativa è un termine reso popolare da Brian Eno per descrivere la musica che è sempre diversa e mutevole e che è creata da un sistema.\\
 Ci sono quattro prospettive primarie sulla musica generativa (Wooller, R. et al., 2005) (riprodotte con permesso):

 \subsubsection*{Linguistico/Strutturale}

 Musica composta da teorie analitiche così esplicite da poter generare materiale strutturalmente coerente (Loy e Abbott 1985; Cope 1991).
 Questa prospettiva ha le sue radici nelle grammatiche generative del linguaggio (Chomsky 1956) e della musica (Lerdahl e Jackendoff 1983), che generano materiale con una struttura ad albero ricorsiva.
 
\subsubsection*{Interattivo/comportamentale}
Musica generata da un componente di sistema che non ha input musicali distinguibili. Cioè, "non trasformativo" (Rowe 1991; Lippe 1997:34; Winkler 1998).
 Il software Wotja di Intermorphic e il software Koan di SSEYO utilizzato da Brian Eno per creare Generative Music 1, sono entrambi esempi di questo approccio.

 \subsubsection*{Creativo/procedurale}
 Musica generata da processi progettati e/o avviati dal compositore. It's Gonna Rain di Steve Reich e In C di Terry Riley ne sono esempi (Eno 1996).

 \subsubsection*{Biologico/emergente}
 Musica non deterministica (Biles 2002), o musica che non può essere ripetuta, ad esempio, i normali campanelli a vento (Dorin 2001). Questa prospettiva proviene dal più ampio movimento dell'arte generativa.\\
 Questo ruota attorno all'idea che la musica, o i suoni, possano essere "generati" da un musicista "coltivando" parametri all'interno di un'ecologia, in modo tale che l'ecologia produrrà perpetuamente variazioni diverse in base ai parametri e agli algoritmi utilizzati.\\ 
 Un esempio di questa tecnica è Viral symphOny di Joseph Nechvatal: una sinfonia collaborativa di musica noise elettronica creata tra gli anni 2006 e 2008 utilizzando un software di vita artificiale personalizzato basato su un modello virale.

\section{Materiali}

\section{Implementazione in Ambisonics}