% !TEX TS-program = pdflatex
% !TEX root = ../tesi.tex

%************************************************
\chapter{ABSTRACT}
\label{chp:abstract}
%************************************************

Questo progetto prevede l'utilizzo della metodologia Ambisonics per il mixing e la produzione tridimensionale dell'audio all'interno di un percorso videoludico 3D, riprodotto mediante l'uso del motore grafico Unity per lo sviluppo del contenuto interattivo e di animazione visiva. \\
Ambisonics di ordine superiore ha trovato un mercato di nicchia nei videogiochi sviluppati da Codemasters. \\
 Il loro primo gioco a utilizzare un motore audio Ambisonic è stato "Colin McRae: DiRT”
Le prime musiche per videogiochi erano generalmente monofoniche, suonate adoperando apparecchiature quali computer e sintetizzatori, e venivano messe in loop, oppure riprodotte fra un livello e l'altro. \\
Alcuni esempi includono i titoli di Pac-Man della Namco (1980) composti da Toshio Kai. \\
 Il primo videogioco a presentare una colonna sonora dotata di un sottofondo continuo fu quella di Space Invaders della Taito Corporation (1978), realizzata da Tomohiro Nishikado. A partire dal 1980, alcuni videogiochi iniziarono ad adoperare apparecchiature digitali, campionamenti e sintesi FM per fare musica. \\
Rally-X della Namco, uscito nel 1980,
fu probabilmente il primo videogame a presentare una colonna sonora realizzata con un convertitore di segnale analogico (DAC) per riprodurre note campionate. \\
Mentre le console casalinghe si avvicinavano alla quarta generazione (o era dei 16-bit),
il compositore Yuzo Koshiro utilizzò l'hardware del Mega Drive per comporre "brani progressivi, techno, e orecchiabili molto evoluti". \\
A partire dagli anni `90, la musica per videogiochi iniziò a presentare sonorità realizzate con strumenti musicali veri e propri. \\
La musica per videogiochi arrivò ad amalgamarsi, infine, coi mondi musicali del pop, hip ed electro. \\
La mia personale aggiunta storica, dal punto di vista musicale, avvalendomi del metodo Ambisonics, sarà quella di inserire una composizione del genere "Computer Music", come novità, all'interno di una rappresentazione videoludica. \\
